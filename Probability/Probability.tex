\documentclass[11pt]{book}
\usepackage{amsmath}
\usepackage{amsfonts}
\usepackage{amssymb}
\usepackage{geometry}
\geometry{a4paper, margin=1in}
\usepackage{graphicx}
\usepackage[hidelinks]{hyperref}
\usepackage{amsthm}
\usepackage{tikz}
\usepackage{subcaption}
\usetikzlibrary{positioning}
\usepackage{pgfplots} 
\usepackage[ruled,vlined]{algorithm2e} 
\usepackage{dsfont}
\usepackage{graphicx}
\usepackage{mathdesign}
\usepackage{float}
\usepackage{todonotes} 
\usepackage{empheq}
\usepackage{array}
\usepackage[ruled,vlined]{algorithm2e} 



\setlength{\parindent}{0pt}


\newcommand\mycommfont[1]{\footnotesize\ttfamily\textcolor{blue}{#1}}
\newcommand\defeq{\stackrel{\mathclap{\normalfont\mbox{def}}}{=}}
\SetCommentSty{mycommfont}

\DeclareMathOperator*{\argmax}{argmax}
\DeclareMathOperator*{\argmin}{argmin}

\newtheorem{theorem}{Theorem}[section]
\newtheorem{lemma}{Lemma}[section]
\newtheorem{definition}{Definition}[section]
\newtheorem{corollary}{Corollary}[section]
\newtheorem{claim}{claim}[section]
\newtheorem{example}{Example}[section]


\newtheorem*{claim*}{Claim}
\newtheorem*{lemma*}{Lemma}
\newtheorem*{corollary*}{Corollary}
\newtheorem*{remark*}{Remark}
\newtheorem*{example*}{Example}
\newtheorem*{examples*}{Examples}
\newtheorem*{definition*}{Definition}



\setcounter{tocdepth}{3}





\begin{document}

\begin{titlepage}
    \begin{center}
     {\huge\bfseries 
     Probability \\
     }
     % ----------------------------------------------------------------
     \vspace{1.5cm}
     {\Large\bfseries Hadar Tal}\\[5pt]
     hadar.tal@mail.huji.ac.il\\[14pt]
      % ----------------------------------------------------------------
     \vspace{2cm}
     {This paper is a summary of the educational materials and lectures from 
     \begin{itemize}
        \item \textbf{Optimization for Computer Science} by Professor Tomer Koren, Tel Aviv University
        \item \textbf{Wikipedia}
        \item \textbf{3Blue1Brown} YouTube channel
        \item Introduction to the Wasserstein distance by "Applied Algebraic Topology Network" YouTube channel
     \end{itemize}
     }

     \vfill
    {Winter 2024}
    \end{center}
\end{titlepage}


\frontmatter
\tableofcontents

% * * * * * * * * * * * * * * * * * * * * * * * * 
% * * * * * * * * * * * * * * * * * * * * * * * * 
% * * * * * * * * * * * * * * * * * * * * * * * * 
% * * * * * * * * * * * * * * * * * * * * * * * * 
% * * * * * * * * * * * * * * * * * * * * * * * * 
% * * * * * * * * * * * * * * * * * * * * * * * * 
% * * * * * * * * * * * * * * * * * * * * * * * * 
% * * * * * * * * * * * * * * * * * * * * * * * * 
% * * * * * * * * * * * * * * * * * * * * * * * * 
% * * * * * * * * * * * * * * * * * * * * * * * * 
% * * * * * * * * * * * * * * * * * * * * * * * * 
% * * * * * * * * * * * * * * * * * * * * * * * * 
% * * * * * * * * * * * * * * * * * * * * * * * * 
% * * * * * * * * * * * * * * * * * * * * * * * * 
% * * * * * * * * * * * * * * * * * * * * * * * * 
% * * * * * * * * * * * * * * * * * * * * * * * * 
% * * * * * * * * * * * * * * * * * * * * * * * * 
% * * * * * * * * * * * * * * * * * * * * * * * * 
% * * * * * * * * * * * * * * * * * * * * * * * * 

\mainmatter


% * * * * * * * * * * * * * * * * * * * * * * * * * * * * * * * * * * * * * * * * * * * * * * * *
% * * * * * * * * * * * * * * * * * * * * * * * * * * * * * * * * * * * * * * * * * * * * * * * *
% * * * * * * * * * * * * * * * * * * * * * * * * * * * * * * * * * * * * * * * * * * * * * * * *
% * * * * * * * * * * * * * * * * * * * * * * * * * * * * * * * * * * * * * * * * * * * * * * * *
% * * * * * * * * * * * * * * * * * * * * * * * * * * * * * * * * * * * * * * * * * * * * * * * *
% * * * * * * * * * * * * * * * * * * * * * * * * * * * * * * * * * * * * * * * * * * * * * * * *
% * * * * * * * * * * * * * * * * * * * * * * * * * * * * * * * * * * * * * * * * * * * * * * * *
% * * * * * * * * * * * * * * * * * * * * * * * * * * * * * * * * * * * * * * * * * * * * * * * *
% * * * * * * * * * * * * * * * * * * * * * * * * * * * * * * * * * * * * * * * * * * * * * * * *
% * * * * * * * * * * * * * * * * * * * * * * * * * * * * * * * * * * * * * * * * * * * * * * * *
% * * * * * * * * * * * * * * * * * * * * * * * * * * * * * * * * * * * * * * * * * * * * * * * *
% * * * * * * * * * * * * * * * * * * * * * * * * * * * * * * * * * * * * * * * * * * * * * * * *
% * * * * * * * * * * * * * * * * * * * * * * * * * * * * * * * * * * * * * * * * * * * * * * * *
% * * * * * * * * * * * * * * * * * * * * * * * * * * * * * * * * * * * * * * * * * * * * * * * *
% * * * * * * * * * * * * * * * * * * * * * * * * * * * * * * * * * * * * * * * * * * * * * * * *
% * * * * * * * * * * * * * * * * * * * * * * * * * * * * * * * * * * * * * * * * * * * * * * * *
% * * * * * * * * * * * * * * * * * * * * * * * * * * * * * * * * * * * * * * * * * * * * * * * *
% * * * * * * * * * * * * * * * * * * * * * * * * * * * * * * * * * * * * * * * * * * * * * * * *
% * * * * * * * * * * * * * * * * * * * * * * * * * * * * * * * * * * * * * * * * * * * * * * * *
% * * * * * * * * * * * * * * * * * * * * * * * * * * * * * * * * * * * * * * * * * * * * * * * *
% * * * * * * * * * * * * * * * * * * * * * * * * * * * * * * * * * * * * * * * * * * * * * * * *

 
\chapter{Probability Measures}

\section{Probability Spaces}

\begin{definition}{\textbf{Probability Space}} \\
    A probability space is a triple \( (\Omega, \mathcal{F}, P) \) where:
    \begin{itemize}
        \item \( \Omega \) is a set of outcomes.
        \item \( \mathcal{F} \) is a sigma-algebra of events.
        \item \( P \) is a probability measure.
    \end{itemize}
\end{definition}

\begin{definition}{\textbf{Sigma-Algebra}} \\
    A sigma-algebra is a collection of subsets of a set \( \Omega \) that contains the empty set, is closed under complements, and is closed under countable unions.
\end{definition}

\begin{definition}{\textbf{Probability Measure}} \\
    A probability measure is a function \( P: \mathcal{F} \rightarrow [0,1] \) that satisfies:
    \begin{itemize}
        \item \( P(\Omega) = 1 \).
        \item For any countable collection of disjoint sets \( A_1, A_2, \ldots \in \mathcal{F} \), 
        \[
        P\left( \bigcup_{i=1}^{\infty} A_i \right) = \sum_{i=1}^{\infty} P(A_i).
        \]
    \end{itemize}
\end{definition}


\chapter{Distances and Metrics}

\section{Wasserstein Distance}

The Wasserstein distance (also called the $\textbf{kanterovich-rubinstein distance}$, \\
$\textbf{optimal transport distance}$, or $\textbf{earth mover's distance}$) 
is a measure of the distance between two probability distributions that has gained popularity in various fields, 
particularly in machine learning for its application in generative models.

\medbreak

While using function distance $\| f - g \|_\infty$ is a common way to measure the difference between two functions,
it is not always the most appropriate. 

\medbreak

\textbf{Intuition:} 
\begin{itemize}
    \item The Wasserstein distance can be thought of as the minimum "cost" required to transform one distribution into another, 
    where the cost is measured by the amount of "mass" that must be moved times the distance it has to be moved.
    \item Amount of work needed to transform one distribution into another.
    \item Area between cumulative distribution functions (CDFs).
    \item Sensitive to outliers.
\end{itemize}

We will begin by looking at the discrete case of a Transport Plan. 

Lets consider two discrete probability distributions \( X = \{x_1, x_2, x_3, x_4\} \) and \( Y = \{y_1, y_2, y_3\} \)
with the probability mass functions:
\[
    P(X) = \{0.4, 0.1, 0.3, 0.2\} \quad \text{and} \quad P(Y) = \{0.3, 0.4, 0.3\}.
\]

We can define a transport plan as a matrix \( \pi \) where \( \pi_{ij} \) represents the amount of mass that is transported from \( x_i \) to \( y_j \).


\[
\begin{array}{c|cccc}
  & x_1 & x_2 & x_3 & x_4 \\
\hline
y_1 & 0.3 & 0.3 & 0   & 0   \\
y_2 & 0.4 & 0.1 & 0.2 & 0   \\
y_3 & 0.3 & 0   & 0.1 & 0.2 \\
\end{array}
\]

\[
d_{W_1} \left( \sum_{i} \alpha_i \delta_{x_i}, \sum_{j} \beta_j \delta_{y_j} \right) = 
\min \{ \sum_{i,j} \pi_{ij} d(x_i, y_j) : \pi_{ij} \geq 0, 
\sum_{i} \pi_{ij} = \beta_j, \sum_{j} \pi_{ij} = \alpha_i \}
\]


\begin{definition}{\textbf{Wasserstein Distance}} \\
    Given two probability measures \( \mu \) and \( \nu \) on a metric space \( (M, d) \), the Wasserstein distance of order \( p \) between \( \mu \) and \( \nu \) is defined as:
    \[
    W_p(\mu, \nu) = \left( \inf_{\gamma \in \Gamma(\mu, \nu)} \int_{M \times M} d(x,y)^p \, d\gamma(x,y) \right)^{\frac{1}{p}},
    \]
    where \( \Gamma(\mu, \nu) \) denotes the collection of all measures on \( M \times M \) with marginals \( \mu \) and \( \nu \) on the first and second factors, respectively.
\end{definition}

And for $W_1$:
\[
    W_1(\mu, \nu) = \inf_{\gamma \in \Gamma(\mu, \nu)} \int_{M \times M} d(x,y) \, d\gamma(x,y).
\]


\textbf{Example:} Consider two one-dimensional distributions, \( \mu \) which is a delta function at 0, 
and \( \nu \) which is a delta function at 1. The Wasserstein distance of order 1 between \( \mu \) and \( \nu \) is simply the distance between the two points, which is 1.

This metric is particularly useful when comparing probability distributions that might not have the same support or when the cost of transportation is significant.


\section{Total Variation Distance}

\section{Kullback-Leibler Divergence}

\section{Hellinger Distance}

\end{document}