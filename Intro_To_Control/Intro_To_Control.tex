\documentclass[a4 paper]{article}
\usepackage[inner=2.0cm,outer=2.0cm,top=2.5cm,bottom=2.5cm]{geometry}
\usepackage{setspace}
\usepackage[rgb]{xcolor}
\usepackage{verbatim}
\usepackage{subcaption}
\usepackage{amsgen,amsmath,amstext,amsbsy,amsopn,tikz,amssymb}
\usepackage{fancyhdr}
\usepackage[colorlinks=true, urlcolor=blue,  linkcolor=blue, citecolor=blue]{hyperref}
\usepackage[colorinlistoftodos]{todonotes}
\usepackage{rotating}
\usepackage{tikz}
\usetikzlibrary{decorations.pathreplacing} 
\usetikzlibrary{positioning}
\usepackage{float}
\usepackage{minted}
\usepackage{amsthm}
\usepackage{amsmath}
\usepackage{amsfonts}
\usepackage{amssymb}
\usepackage{geometry}
\geometry{a4paper, margin=1in}
\usepackage{graphicx}
% \usepackage[hidelinks]{hyperref}
\usepackage{dsfont}
\usepackage{graphicx}
\usepackage{mathdesign}
\usepackage{float}
\usepackage{todonotes} 
\usepackage{empheq}
\usepackage{array}
\usepackage[ruled,vlined]{algorithm2e} 
\usepackage[many]{tcolorbox}    	% for COLORED BOXES (tikz and xcolor included)
\usepackage{bookmark}

\usepackage{pgfplots} 
\pgfplotsset{compat=1.18}


\newtcolorbox{boxA}{
    fontupper = \bf,
    boxrule = 1.5pt,
    colframe = black % frame color
}


\setlength{\parindent}{0pt}
\numberwithin{equation}{section}


\newcommand\mycommfont[1]{\footnotesize\ttfamily\tblu{#1}}
\newcommand\defeq{\stackrel{\mathclap{\normalfont\mbox{def}}}{=}}
\SetCommentSty{mycommfont}

\DeclareMathOperator*{\argmax}{argmax}
\DeclareMathOperator*{\argmin}{argmin}



\newtheoremstyle{boldStyle}%                % Name
  {}%                                     % Space above
  {}%                                     % Space below
  {\itshape}%                                     % Body font
  {}%                                     % Indent amount
  {\bfseries}%                            % Theorem head font
  {}%                                    % Punctuation after theorem head
  {\newline}                              % Space after theorem head, new line
  {\thmname{#1}\thmnumber{ #2}\thmnote{ (#3)}}%                                     % Theorem head spec (can be left empty, meaning `normal')


\theoremstyle{boldStyle}
\newtheorem{example}{Example}[section]
\newtheorem{question}{Question}[section]
\newtheorem{answer}{Answer}[section]


\newtheorem*{claim*}{Claim}
\newtheorem*{lemma*}{Lemma}
\newtheorem*{corollary*}{Corollary}
\newtheorem*{remark*}{Remark}
\newtheorem*{example*}{Example}
\newtheorem*{examples*}{Examples}
\newtheorem*{definition*}{Definition}
\newtheorem*{question*}{Question}
\newtheorem*{answer*}{Answer}


\newtheoremstyle{boldBlueStyle}%                % Name
  {}%                                          % Space above
  {}%                                          % Space below
  {\itshape}%                                  % Body font
  {}%                                          % Indent amount
  {\color{blueColor}\bfseries}%          % Theorem head font in red
  {}%                                          % Punctuation after theorem head
  {\newline}%                                  % Space after theorem head, new line
  {\thmname{#1}\thmnumber{ #2}\thmnote{ (#3)}} % Theorem head spec (can be left empty, meaning `normal')

\theoremstyle{boldBlueStyle}
\newtheorem{lemma}{Lemma}[section]
\newtheorem{corollary}{Corollary}[section]
\newtheorem{claim}{claim}[section]
\newtheorem{proposition}{Proposition}[section]



\newtheoremstyle{boldPurpleStyle}%                % Name
  {}%                                          % Space above
  {}%                                          % Space below
  {\itshape}%                                  % Body font
  {}%                                          % Indent amount
  {\color{purpleColor}\bfseries}%          % Theorem head font in red
  {}%                                          % Punctuation after theorem head
  {\newline}%                                  % Space after theorem head, new line
  {\thmname{#1}\thmnumber{ #2}\thmnote{ (#3)}} % Theorem head spec (can be left empty, meaning `normal')

\theoremstyle{boldPurpleStyle}
\newtheorem{theorem}{Theorem}[section]


\newtheoremstyle{boldRedStyle}%                % Name
  {}%                                          % Space above
  {}%                                          % Space below
  {\itshape}%                                  % Body font
  {}%                                          % Indent amount
  {\color{redColor}\bfseries}%          % Theorem head font in red
  {}%                                          % Punctuation after theorem head
  {\newline}%                                  % Space after theorem head, new line
  {\thmname{#1}\thmnumber{ #2}\thmnote{ (#3)}} % Theorem head spec (can be left empty, meaning `normal')

\theoremstyle{boldRedStyle}
\newtheorem{definition}{Definition}[section]




%\usetikzlibrary{through,backgrounds}
\hypersetup{%
pdfauthor={Ashudeep Singh},%
pdftitle={Homework},%
pdfkeywords={Tikz,latex,bootstrap,uncertaintes},%
pdfcreator={PDFLaTeX},%
pdfproducer={PDFLaTeX},%
}
%\usetikzlibrary{shadows}
% \usepackage[francais]{babel}
\usepackage{booktabs}
\input{../Latex_Utils/macros.tex}

\setlength{\parindent}{0pt}


\begin{document}
\homework{67678 - Introduction to Control with Learning}{Linear Dynamical Systems - Summary}{Spring 2024}{Prof. Oron Sabag}{}{Hadar Tal}{}


\section{Kalman Filter}


\begin{definition}[Kalman Filter State Space] 
  \begin{equation}
    \begin{aligned}
      x_{i+1} &= \tred{F} x_i + \tred{G} w_i \\
      y_i &= \tred{H} x_i + v_i
    \end{aligned}
  \end{equation}

  where:
\begin{itemize}
    \item \( x_{i+1} \) is the state vector at time \( i+1 \),
    \item \( x_i \) is the state vector at time \( i \),
    \item \( y_i \) is the measurement vector at time \( i \),
    \item \( w_i \) is the process noise (zero mean, uncorrelated),
    \item \( v_i \) is the measurement noise (zero mean, uncorrelated).
\end{itemize}
\end{definition}


\begin{definition}[Kalman Filter Covariance Matrix] 
  Formally, the following covariance matrix describes the model:
  \begin{equation}
      \mathbb{E} \left[
      \begin{pmatrix}
          w_i \\
          v_i \\
          x_0
      \end{pmatrix}
      \begin{pmatrix}
          w_j^* & v_j^* & x_0^* & 1
      \end{pmatrix}
      \right] = 
      \begin{pmatrix}
          \begin{pmatrix}
            \tblu{Q} & \tblu{S} \\
            \tblu{S^*} & \tblu{R}
          \end{pmatrix} \delta_{ij} & 0 & 0 \\
          0 & \tblu{\Pi_0} & 0
      \end{pmatrix},
      \label{eq:covariance_matrix}
  \end{equation}
  where \(
    \begin{pmatrix}
      \tblu{Q} & \tblu{S} \\
      \tblu{S^*} & \tblu{R}
    \end{pmatrix}
  \) and \(\tblu{\Pi_0}\) are positive semidefinite matrices and \(\delta_{ij}\) equals 1 if \(i = j\) and is zero otherwise.
  
  Note that \(w_i\) is uncorrelated as a process over time but its coordinates at a fixed time can be correlated via \(Q\).
\end{definition}

\textbf{Markings:}
\begin{itemize}
    \item \tpur{\( P_i \)} - The error covariance matrix at time \( i \)
    \begin{equation}
        \tpur{P_i} \eqdef (x_i - \hat{x}_i)(x_i - \hat{x}_i)^T
    \end{equation}

    \item \tpur{\( R_{e,i} \)} - The covariance of the innovation (or residual) at time \( i \)
    \begin{equation}
        \tpur{R_{e,i}} \eqdef \tred{H} \tpur{P_i} \tred{H^*} + \tblu{R}  
    \end{equation}

    \item \(\tpur{K_{p,i}}\) - The optimal Kalman gain at time \( i \)
    \begin{equation}
        \tpur{K_{p,i}} \triangleq (\tred{F} \tpur{P_i} \tred{H^*} + \tred{G}\tblu{S}) \tpur{R_{e,i}}^{-1}
    \end{equation}

\end{itemize}

\subsection*{Kalman Filter Optimality}

We suggest the following predictor:
\begin{equation}
    \hat{x}_{i+1|i} = \tred{F} \hat{x}_{i|i-1} + \tpur{K_{p,i}} (y_i - \tred{H} \hat{x}_{i|i-1})
\end{equation}

\begin{lemma}
  \begin{equation}
    \tilde{x}_{i+1} = (F - K_{p,i} H) \tilde{x}_i + (G - K_{p,i}) 
    \begin{pmatrix}
    w_i \\
    v_i
    \end{pmatrix}.
    \end{equation}
\end{lemma}

\begin{proof}
\begin{align*}
\tilde{x}_{i+1} &= x_{i+1} - \hat{x}_{i+1|i} \\
&= (F x_i + G w_i) - \left( F \hat{x}_i + K_{p,i} (y_i - H \hat{x}_i) \right) \\
&= F x_i + G w_i - F \hat{x}_i - K_{p,i} (H x_i + v_i - H \hat{x}_i) \\
&= F x_i + G w_i - F \hat{x}_i - K_{p,i} H x_i - K_{p,i} v_i + K_{p,i} H \hat{x}_i \\
&= F x_i - F \hat{x}_i - K_{p,i} H x_i + K_{p,i} H \hat{x}_i + G w_i - K_{p_i} v_i \\
&= (F - K_{p,i} H)(x_i - \hat{x}_i) + G w_i - K_{p,i} v_i \\
&= (F - K_{p,i} H) \tilde{x}_i + G w_i - K_{p,i} v_i.
\end{align*}
\end{proof}


\begin{lemma}
  For \( j < i \), the recursion can be evolved as
  \begin{align*}
      \tilde{x}_i &= (F - K_{p,i-1}H) \tilde{x}_{i-1} + (G - K_{p,i-1})
      \begin{pmatrix}
          w_{i-1} \\
          v_{i-1}
      \end{pmatrix} \\
      &= \ldots \\
      &= \phi_p(i, j) \tilde{x}_j + \xi_i(j),
  \end{align*}
  where
  \begin{align*}
      \phi_p(i, j) &= \prod_{k=j}^{i-1} (F - K_{p,k}H), \\
      \xi_i(j) &= \sum_{k=j}^{i-1} \phi_p(i, k+1) (G w_k - K_{p,k} v_k).
  \end{align*}
  \end{lemma}  

\end{document}