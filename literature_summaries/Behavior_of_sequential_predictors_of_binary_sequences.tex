\documentclass[11pt]{article} 
\usepackage{amsmath}
\usepackage{amsfonts}
\usepackage{amssymb}
\usepackage{geometry}
\geometry{a4paper, margin=1in}
\usepackage{graphicx}
\usepackage[hidelinks]{hyperref}
\usepackage{amsthm}
\usepackage{tikz}
\usepackage{subcaption}
\usetikzlibrary{positioning}
\usepackage{pgfplots} 
\usepackage[ruled,vlined]{algorithm2e} 
\usepackage{dsfont}
\usepackage{graphicx}
\usepackage{mathdesign}
\usepackage{float}
\usepackage{todonotes} 
\usepackage{empheq}
\usepackage{array}
\usepackage[ruled,vlined]{algorithm2e} 
\usepackage[many]{tcolorbox}    	% for COLORED BOXES (tikz and xcolor included)



\newtcolorbox{boxA}{
    fontupper = \bf,
    boxrule = 1.5pt,
    colframe = black % frame color
}


\setlength{\parindent}{0pt}

% \let\stdsection\section
% \renewcommand\section{\newpage\stdsection}

\newcommand\mycommfont[1]{\footnotesize\ttfamily\textcolor{blue}{#1}}
\newcommand\defeq{\stackrel{\mathclap{\normalfont\mbox{def}}}{=}}
\SetCommentSty{mycommfont}

\DeclareMathOperator*{\argmax}{argmax}
\DeclareMathOperator*{\argmin}{argmin}



\newtheoremstyle{boldStyle}%                % Name
  {}%                                     % Space above
  {}%                                     % Space below
  {\itshape}%                                     % Body font
  {}%                                     % Indent amount
  {\bfseries}%                            % Theorem head font
  {}%                                    % Punctuation after theorem head
  {\newline}                              % Space after theorem head, new line
  {\thmname{#1}\thmnumber{ #2}\thmnote{ (#3)}}%                                     % Theorem head spec (can be left empty, meaning `normal')


\theoremstyle{boldStyle}
\newtheorem{theorem}{Theorem}[section]
\newtheorem{lemma}{Lemma}[section]
\newtheorem{definition}{Definition}[section]
\newtheorem{corollary}{Corollary}[section]
\newtheorem{claim}{claim}[section]
\newtheorem{example}{Example}[section]


\newtheorem*{claim*}{Claim}
\newtheorem*{lemma*}{Lemma}
\newtheorem*{corollary*}{Corollary}
\newtheorem*{remark*}{Remark}
\newtheorem*{example*}{Example}
\newtheorem*{examples*}{Examples}
\newtheorem*{definition*}{Definition}



\title{
    \Huge Transactions of the Fourth Prague Conference\\
    \Large on Information Theory, Statistical Decision Functions, Random Processes \\
    \large Held at Prague, from August 31 to September 11, 1965 \\
    \vspace{10pt}
    \normalsize Behaviour of Sequential Predictors of Binary Sequences
}

\author{T. M. Cover}
\date{Published by the Publishing House of the Czechoslovak Academy of Sciences, Prague 1967}



\begin{document}
\maketitle

% * * * * * * * * * * * * * * * * * * * * * * * * 
% * * * * * * * * * * * * * * * * * * * * * * * * 
% * * * * * * * * * * * * * * * * * * * * * * * * 
% * * * * * * * * * * * * * * * * * * * * * * * * 
% * * * * * * * * * * * * * * * * * * * * * * * * 
% * * * * * * * * * * * * * * * * * * * * * * * * 
% * * * * * * * * * * * * * * * * * * * * * * * * 
% * * * * * * * * * * * * * * * * * * * * * * * * 
% * * * * * * * * * * * * * * * * * * * * * * * * 
% * * * * * * * * * * * * * * * * * * * * * * * * 
% * * * * * * * * * * * * * * * * * * * * * * * * 
% * * * * * * * * * * * * * * * * * * * * * * * * 
% * * * * * * * * * * * * * * * * * * * * * * * * 
% * * * * * * * * * * * * * * * * * * * * * * * * 
% * * * * * * * * * * * * * * * * * * * * * * * * 
% * * * * * * * * * * * * * * * * * * * * * * * * 
% * * * * * * * * * * * * * * * * * * * * * * * * 
% * * * * * * * * * * * * * * * * * * * * * * * * 
% * * * * * * * * * * * * * * * * * * * * * * * * 
\section{Introduction}


% * * * * * * * * * * * * * * * * * * * * * * * * 
% * * * * * * * * * * * * * * * * * * * * * * * * 
% * * * * * * * * * * * * * * * * * * * * * * * * 
% * * * * * * * * * * * * * * * * * * * * * * * * 
% * * * * * * * * * * * * * * * * * * * * * * * * 
% * * * * * * * * * * * * * * * * * * * * * * * * 
% * * * * * * * * * * * * * * * * * * * * * * * * 
% * * * * * * * * * * * * * * * * * * * * * * * * 
% * * * * * * * * * * * * * * * * * * * * * * * * 
% * * * * * * * * * * * * * * * * * * * * * * * * 
% * * * * * * * * * * * * * * * * * * * * * * * * 
% * * * * * * * * * * * * * * * * * * * * * * * * 
% * * * * * * * * * * * * * * * * * * * * * * * * 
% * * * * * * * * * * * * * * * * * * * * * * * * 
% * * * * * * * * * * * * * * * * * * * * * * * * 
% * * * * * * * * * * * * * * * * * * * * * * * * 
% * * * * * * * * * * * * * * * * * * * * * * * * 
% * * * * * * * * * * * * * * * * * * * * * * * * 
% * * * * * * * * * * * * * * * * * * * * * * * * 
\section{Deterministic Predictors}

Consider the set of $2^n$ sequences $\Theta = (\Theta_1, \Theta_2, \ldots, \Theta_{n}) \in {0, 1}^n$.

At stage $k$, after the observation $\Theta_1, \Theta_2, \ldots, \Theta_{k-1}$, the prediction 1 or 0 will be made with probability $p_k$ and $1-p_k$ respectively.

\begin{definition}[Sequential Predictor]
    A sequential predictor is a sequence of functions $p_1, p_2, \ldots, p_n$ taking values in $[0, 1]$.
\end{definition}

Thus, a sequential predictor on ${0, 1}^n$ will be completely specified by the set of functions 
$p_1, p_2(\Theta_1), p_3(\Theta_1, \Theta_2), \ldots, p_n(\Theta_1, \Theta_2, \ldots, \Theta_{n-1})$ 
taking values in $[0, 1]$.

% * * * * * * * * * * * * * * * * * * * * * * * * 
% * * * * * * * * * * * * * * * * * * * * * * * * 
% * * * * * * * * * * * * * * * * * * * * * * * * 
% * * * * * * * * * * * * * * * * * * * * * * * * 
% * * * * * * * * * * * * * * * * * * * * * * * * 
% * * * * * * * * * * * * * * * * * * * * * * * * 
% * * * * * * * * * * * * * * * * * * * * * * * * 
% * * * * * * * * * * * * * * * * * * * * * * * * 
% * * * * * * * * * * * * * * * * * * * * * * * * 
% * * * * * * * * * * * * * * * * * * * * * * * * 
% * * * * * * * * * * * * * * * * * * * * * * * * 
% * * * * * * * * * * * * * * * * * * * * * * * * 
% * * * * * * * * * * * * * * * * * * * * * * * * 
% * * * * * * * * * * * * * * * * * * * * * * * * 
% * * * * * * * * * * * * * * * * * * * * * * * * 
% * * * * * * * * * * * * * * * * * * * * * * * * 
% * * * * * * * * * * * * * * * * * * * * * * * * 
% * * * * * * * * * * * * * * * * * * * * * * * * 
% * * * * * * * * * * * * * * * * * * * * * * * * 
\section{Sequential Betting Systems}




% * * * * * * * * * * * * * * * * * * * * * * * * 
% * * * * * * * * * * * * * * * * * * * * * * * * 
% * * * * * * * * * * * * * * * * * * * * * * * * 
% * * * * * * * * * * * * * * * * * * * * * * * * 
% * * * * * * * * * * * * * * * * * * * * * * * * 
% * * * * * * * * * * * * * * * * * * * * * * * * 
% * * * * * * * * * * * * * * * * * * * * * * * * 
% * * * * * * * * * * * * * * * * * * * * * * * * 
% * * * * * * * * * * * * * * * * * * * * * * * * 
% * * * * * * * * * * * * * * * * * * * * * * * * 
% * * * * * * * * * * * * * * * * * * * * * * * * 
% * * * * * * * * * * * * * * * * * * * * * * * * 
% * * * * * * * * * * * * * * * * * * * * * * * * 
% * * * * * * * * * * * * * * * * * * * * * * * * 
% * * * * * * * * * * * * * * * * * * * * * * * * 
% * * * * * * * * * * * * * * * * * * * * * * * * 
% * * * * * * * * * * * * * * * * * * * * * * * * 
% * * * * * * * * * * * * * * * * * * * * * * * * 
% * * * * * * * * * * * * * * * * * * * * * * * * 
\section{Random Predictors}





% * * * * * * * * * * * * * * * * * * * * * * * * 
% * * * * * * * * * * * * * * * * * * * * * * * * 
% * * * * * * * * * * * * * * * * * * * * * * * * 
% * * * * * * * * * * * * * * * * * * * * * * * * 
% * * * * * * * * * * * * * * * * * * * * * * * * 
% * * * * * * * * * * * * * * * * * * * * * * * * 
% * * * * * * * * * * * * * * * * * * * * * * * * 
% * * * * * * * * * * * * * * * * * * * * * * * * 
% * * * * * * * * * * * * * * * * * * * * * * * * 
% * * * * * * * * * * * * * * * * * * * * * * * * 
% * * * * * * * * * * * * * * * * * * * * * * * * 
% * * * * * * * * * * * * * * * * * * * * * * * * 
% * * * * * * * * * * * * * * * * * * * * * * * * 
% * * * * * * * * * * * * * * * * * * * * * * * * 
% * * * * * * * * * * * * * * * * * * * * * * * * 
% * * * * * * * * * * * * * * * * * * * * * * * * 
% * * * * * * * * * * * * * * * * * * * * * * * * 
% * * * * * * * * * * * * * * * * * * * * * * * * 
% * * * * * * * * * * * * * * * * * * * * * * * * 
\section{Conclusions}




\end{document}