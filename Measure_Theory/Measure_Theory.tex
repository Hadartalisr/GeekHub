\documentclass[11pt]{book} 
\usepackage{amsmath}
\usepackage{amsfonts}
\usepackage{amssymb}
\usepackage{geometry}
\geometry{a4paper, margin=1in}
\usepackage{graphicx}
\usepackage[hidelinks]{hyperref}
\usepackage{amsthm}
\usepackage{tikz}
\usepackage{subcaption}
\usetikzlibrary{positioning}
\usepackage{pgfplots} 
\usepackage[ruled,vlined]{algorithm2e} 
\usepackage{dsfont}
\usepackage{graphicx}
\usepackage{mathdesign}
\usepackage{float}
\usepackage{todonotes} 
\usepackage{empheq}
\usepackage{array}
\usepackage[ruled,vlined]{algorithm2e} 
\usepackage[many]{tcolorbox}    	% for COLORED BOXES (tikz and xcolor included)



\newtcolorbox{bluebox}{
    fontupper = \bf,
    boxrule = 1.5pt,
    colframe = black,
    colback = blue!30
}

\newtcolorbox{redbox}{
    fontupper = \bf,
    boxrule = 1.5pt,
    colframe = black,
    colback = red!50
}

\newtcolorbox{purplebox}{
    fontupper = \bf,
    boxrule = 1.5pt,
    colframe = black,
    colback = purple!50
}


\setlength{\parindent}{0pt}
\numberwithin{equation}{chapter}


\newcommand\mycommfont[1]{\footnotesize\ttfamily\textcolor{blue}{#1}}
\newcommand\defeq{\stackrel{\mathclap{\normalfont\mbox{def}}}{=}}
\SetCommentSty{mycommfont}

\DeclareMathOperator*{\argmax}{argmax}
\DeclareMathOperator*{\argmin}{argmin}



\newtheoremstyle{boldStyle}%                % Name
  {}%                                     % Space above
  {}%                                     % Space below
  {\itshape}%                                     % Body font
  {}%                                     % Indent amount
  {\bfseries}%                            % Theorem head font
  {}%                                    % Punctuation after theorem head
  {\newline}                              % Space after theorem head, new line
  {\thmname{#1}\thmnumber{ #2}\thmnote{ (#3)}}%                                     % Theorem head spec (can be left empty, meaning `normal')


\theoremstyle{boldStyle}
\newtheorem{theorem}{Theorem}[section]
\newtheorem{lemma}{Lemma}[section]
\newtheorem{definition}{Definition}[section]
\newtheorem{corollary}{Corollary}[section]
\newtheorem{claim}{claim}[section]
\newtheorem{example}{Example}[section]


\newtheorem*{claim*}{Claim}
\newtheorem*{lemma*}{Lemma}
\newtheorem*{corollary*}{Corollary}
\newtheorem*{remark*}{Remark}
\newtheorem*{example*}{Example}
\newtheorem*{examples*}{Examples}
\newtheorem*{definition*}{Definition}



\title{
    \huge Measure Theory\\
    \vspace{10pt}
}

\author{Hadar Tal}

\date{Hebrew University of Jerusalem, Israel, 2024}


\begin{document}
\maketitle

\chapter{Measure Theory}

\section{$\sigma$-algebras and measures}

\begin{definition}($\sigma$-algebra) \\
    Let $\Omega$ be a set. A collection $\mathcal{F}$ of subsets of $\Omega$ is called a $\sigma$-algebra if:
    \begin{enumerate}
        \item $\Omega \in \mathcal{F}$.
        \item If $A \in \mathcal{F}$, then $A^c \in \mathcal{F}$.
        \item If $A_1, A_2, \ldots \in \mathcal{F}$, then $\bigcup_{i=1}^{\infty} A_i \in \mathcal{F}$.
    \end{enumerate}
\end{definition}


\begin{definition}(Measure) \\
    Let $\Omega$ be a set and $\mathcal{F}$ be a $\sigma$-algebra on $\Omega$. A function $\mu: \mathcal{F} \to [0, \infty]$ is called a measure if:
    \begin{enumerate}
        \item $\mu(\emptyset) = 0$.
        \item If $A_1, A_2, \ldots \in \mathcal{F}$ are pairwise disjoint, then $\mu\left(\bigcup_{i=1}^{\infty} A_i\right) = \sum_{i=1}^{\infty} \mu(A_i)$.
    \end{enumerate}
\end{definition}

\begin{definition}(Measure Space) \\
    A measure space is a triple $(\Omega, \mathcal{F}, \mu)$, where $\Omega$ is a set, $\mathcal{F}$ 
    is a $\sigma$-algebra on $\Omega$, and $\mu$ is a measure on $\mathcal{F}$.
\end{definition}


\section{Order of spaces}

\subsection{1. Topological Space (Most General)}

\textbf{Axioms:} A topological space is a set \( X \) together with a collection \( \mathcal{T} \) of subsets of \( X \) (called open sets) such that:
\begin{enumerate}
    \item The empty set and \( X \) itself are in \( \mathcal{T} \).
    \item Any union of sets in \( \mathcal{T} \) is also in \( \mathcal{T} \).
    \item Any finite intersection of sets in \( \mathcal{T} \) is also in \( \mathcal{T} \).
\end{enumerate}

\textbf{Why it's more general:} Topological spaces are the most general because they only require a definition of open sets, without any need for distances or measures. They provide a flexible framework for discussing continuity, compactness, and connectedness.

\vspace{10pt}

\subsection*{2. Measurable Space}

\textbf{Axioms:} A measurable space is a set \( X \) together with a collection \( \mathcal{F} \) of subsets of \( X \) (called measurable sets) such that:
\begin{enumerate}
    \item The empty set and \( X \) itself are in \( \mathcal{F} \).
    \item \( \mathcal{F} \) is closed under complements: if \( A \in \mathcal{F} \), then \( X \setminus A \in \mathcal{F} \).
    \item \( \mathcal{F} \) is closed under countable unions: if \( A_1, A_2, \ldots \in \mathcal{F} \), then \( \bigcup_{i=1}^\infty A_i \in \mathcal{F} \).
\end{enumerate}

\textbf{Why it's less general than a topological space:} Measurable spaces are less general because they require the collection of measurable sets to be closed under complements and countable unions, forming a \(\sigma\)-algebra. This imposes more structure compared to the open sets in a topological space, which only need to be closed under arbitrary unions and finite intersections.

\vspace{10pt}

\subsection*{3. Metric Space (Most Specific)}

\textbf{Axioms:} A metric space is a set \( X \) together with a function \( d: X \times X \to \mathbb{R} \) (called a metric) such that:
\begin{enumerate}
    \item \( d(x, y) \geq 0 \) for all \( x, y \in X \) (non-negativity).
    \item \( d(x, y) = 0 \) if and only if \( x = y \) (identity of indiscernibles).
    \item \( d(x, y) = d(y, x) \) for all \( x, y \in X \) (symmetry).
    \item \( d(x, z) \leq d(x, y) + d(y, z) \) for all \( x, y, z \in X \) (triangle inequality).
\end{enumerate}

\textbf{Why it's less general than a measurable space:} Metric spaces are less general because they require the existence of a metric that defines distances between points. This metric induces a topology, where open sets are defined in terms of open balls around points. While every metric space is a topological space (with the topology induced by the metric), not every topological space can be given a metric that defines its open sets. Furthermore, metric spaces do not inherently involve measurable sets or \(\sigma\)-algebras.


\section{Borel $\sigma$-algebra}


\begin{definition}(Borel \(\sigma\)-algebra) \\
    Let \(X\) be a topological space. The Borel \(\sigma\)-algebra on \(X\), denoted by \(\mathcal{B}(X)\), is the smallest \(\sigma\)-algebra that contains all the open sets of \(X\). Formally, \(\mathcal{B}(X)\) is the \(\sigma\)-algebra generated by the collection of open sets \(\mathcal{T}\) in \(X\):
    \[
    \mathcal{B}(X) = \sigma(\mathcal{T}),
    \]
    where \(\sigma(\mathcal{T})\) denotes the \(\sigma\)-algebra generated by \(\mathcal{T}\). 
    This means that \(\mathcal{B}(X)\) is the smallest collection of subsets of \(X\) that contains all open sets and is
    a \(\sigma\)-algebra.
\end{definition}

Examples of Borel \(\sigma\)-algebras include:

\begin{itemize}
    \item \textbf{Borel \(\sigma\)-algebra on \(\mathbb{R}\)}: The Borel \(\sigma\)-algebra on the real line \(\mathbb{R}\), denoted by \(\mathcal{B}(\mathbb{R})\), is the \(\sigma\)-algebra generated by all open intervals in \(\mathbb{R}\). This includes:
    \begin{itemize}
        \item All open intervals \((a, b)\) where \(a < b\).
        \item All closed intervals \([a, b]\) where \(a < b\).
        \item All half-open intervals \((a, b]\) and \([a, b)\) where \(a < b\).
        \item All singletons \(\{a\}\) where \(a \in \mathbb{R}\).
        \item All countable unions and intersections of such sets.
    \end{itemize}

    \item \textbf{Borel \(\sigma\)-algebra on \(\mathbb{R}^n\)}: The Borel \(\sigma\)-algebra on \(\mathbb{R}^n\), denoted by \(\mathcal{B}(\mathbb{R}^n)\), is the \(\sigma\)-algebra generated by all open sets in the Euclidean space \(\mathbb{R}^n\). This includes:
    \begin{itemize}
        \item All open balls \(B(x, r) = \{ y \in \mathbb{R}^n \mid \|y - x\| < r \}\) where \(x \in \mathbb{R}^n\) and \(r > 0\).
        \item All open rectangles \((a_1, b_1) \times (a_2, b_2) \times \cdots \times (a_n, b_n)\) where \(a_i < b_i\) for each \(i\).
        \item All closed sets and other sets formed by countable unions and intersections of open sets in \(\mathbb{R}^n\).
    \end{itemize}
\end{itemize}



\section{Measurable Functions}

\begin{definition}(Measurable Function) \\
    Let \((\Omega_1, \mathcal{A}_1)\) and \((\Omega_2, \mathcal{A}_2)\) be measurable spaces. 
    A function \(f: \Omega_1 \to \Omega_2\) is called measurable (with respect to \(\mathcal{A}_1\) and \(\mathcal{A}_2\)) if:
    \[
    f^{-1}(A_2) \in \mathcal{A}_1 \quad \text{for all } A_2 \in \mathcal{A}_2.
    \]
\end{definition}

\subsection{Examples}

\begin{enumerate}
    \item \((\Omega, \mathcal{A})\), \((\mathbb{R}, \mathcal{B}(\mathbb{R}))\)
    \begin{itemize}
        \item \textbf{Characteristic Function}:
        \[
        \chi_A: \Omega \to \mathbb{R},
        \]
        where
        \[
        \chi_A(\omega) =
        \begin{cases}
        1, & \omega \in A \\
        0, & \omega \notin A
        \end{cases}
        \]
        For all measurable \(A \in \mathcal{A}\), \(\chi_A\) is a measurable map.
        \begin{itemize}
            \item \(\chi_A^{-1}(\varnothing) = \varnothing \subset \Omega\)
            \item \(\chi_A^{-1}(\mathbb{R}) = \Omega \in A\)
            \item \(\chi_A^{-1}({1}) = A \in \mathcal{A}\)
            \item \(\chi_A^{-1}({0}) = A^c \in \mathcal{A}\) by the closure of \(\mathcal{A}\) under complements.
        \end{itemize}
    \end{itemize}

    \item \((\Omega_1, \mathcal{A}_1)\), \((\Omega_2, \mathcal{A}_2)\), \((\Omega_3, \mathcal{A}_3)\) are measurable spaces
    \begin{itemize}
        \item If \(f: \Omega_1 \to \Omega_2\) is measurable and \(g: \Omega_2 \to \Omega_3\) is measurable, then \(g \circ f: \Omega_1 \to \Omega_3\) is measurable.
        \[
        (g \circ f)^{-1}(A_3) = f^{-1}(g^{-1}(A_3)) \in \mathcal{A}_1 \quad \text{for all } A_3 \in \mathcal{A}_3
        \]
    \end{itemize}
\end{enumerate}

\subsection*{Important}

\((\Omega, \mathcal{A})\), \((\mathbb{R}, \mathcal{B}(\mathbb{R}))\)
\begin{itemize}
    \item \(f, g\) measurable \(\Rightarrow f + g, f \cdot g\) measurable
    \item \(|f|\) measurable
\end{itemize}

\end{document}


% * * * * * * * * * * * * * * * * * * * * * * * * 
% * * * * * * * * * * * * * * * * * * * * * * * * 
% * * * * * * * * * * * * * * * * * * * * * * * * 
% * * * * * * * * * * * * * * * * * * * * * * * * 
% * * * * * * * * * * * * * * * * * * * * * * * * 
% * * * * * * * * * * * * * * * * * * * * * * * * 
% * * * * * * * * * * * * * * * * * * * * * * * * 
% * * * * * * * * * * * * * * * * * * * * * * * * 
% * * * * * * * * * * * * * * * * * * * * * * * * 
% * * * * * * * * * * * * * * * * * * * * * * * * 
% * * * * * * * * * * * * * * * * * * * * * * * * 
% * * * * * * * * * * * * * * * * * * * * * * * * 
% * * * * * * * * * * * * * * * * * * * * * * * * 
% * * * * * * * * * * * * * * * * * * * * * * * * 
% * * * * * * * * * * * * * * * * * * * * * * * * 
% * * * * * * * * * * * * * * * * * * * * * * * * 
% * * * * * * * * * * * * * * * * * * * * * * * * 
% * * * * * * * * * * * * * * * * * * * * * * * * 
% * * * * * * * * * * * * * * * * * * * * * * * * 
% \section*{Summary}




% * * * * * * * * * * * * * * * * * * * * * * * * 
% * * * * * * * * * * * * * * * * * * * * * * * * 
% * * * * * * * * * * * * * * * * * * * * * * * * 
% * * * * * * * * * * * * * * * * * * * * * * * * 
% * * * * * * * * * * * * * * * * * * * * * * * * 
% * * * * * * * * * * * * * * * * * * * * * * * * 
% * * * * * * * * * * * * * * * * * * * * * * * * 
% * * * * * * * * * * * * * * * * * * * * * * * * 
% * * * * * * * * * * * * * * * * * * * * * * * * 
% * * * * * * * * * * * * * * * * * * * * * * * * 
% * * * * * * * * * * * * * * * * * * * * * * * * 
% * * * * * * * * * * * * * * * * * * * * * * * * 
% * * * * * * * * * * * * * * * * * * * * * * * * 
% * * * * * * * * * * * * * * * * * * * * * * * * 
% * * * * * * * * * * * * * * * * * * * * * * * * 
% * * * * * * * * * * * * * * * * * * * * * * * * 
% * * * * * * * * * * * * * * * * * * * * * * * * 
% * * * * * * * * * * * * * * * * * * * * * * * * 
% * * * * * * * * * * * * * * * * * * * * * * * * 



\end{document}