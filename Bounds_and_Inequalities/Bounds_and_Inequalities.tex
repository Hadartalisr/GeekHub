\documentclass[11pt]{book} % or report
\usepackage{amsmath}
\usepackage{amsfonts}
\usepackage{amssymb}
\usepackage{geometry}
\geometry{a4paper, margin=1in}
\usepackage{graphicx}
\usepackage[hidelinks]{hyperref}
\usepackage{amsthm}
\usepackage{tikz}
\usepackage{subcaption}
\usetikzlibrary{positioning}
\usepackage{pgfplots} 
\usepackage[ruled,vlined]{algorithm2e} 
\usepackage{dsfont}
\usepackage{graphicx}
\usepackage{mathdesign}
\usepackage{float}
\usepackage{todonotes} 
\usepackage{empheq}
\usepackage{array}
\usepackage[ruled,vlined]{algorithm2e} 



\setlength{\parindent}{0pt}

% \let\stdsection\section
% \renewcommand\section{\newpage\stdsection}

\newcommand\mycommfont[1]{\footnotesize\ttfamily\textcolor{blue}{#1}}
\newcommand\defeq{\stackrel{\mathclap{\normalfont\mbox{def}}}{=}}
\SetCommentSty{mycommfont}

\DeclareMathOperator*{\argmax}{argmax}
\DeclareMathOperator*{\argmin}{argmin}

\newtheorem{theorem}{Theorem}[section]
\newtheorem{lemma}{Lemma}[section]
\newtheorem{definition}{Definition}[section]
\newtheorem{corollary}{Corollary}[section]
\newtheorem{claim}{claim}[section]
\newtheorem{example}{Example}[section]


\newtheorem*{claim*}{Claim}
\newtheorem*{lemma*}{Lemma}
\newtheorem*{corollary*}{Corollary}
\newtheorem*{remark*}{Remark}
\newtheorem*{example*}{Example}
\newtheorem*{examples*}{Examples}
\newtheorem*{definition*}{Definition}



\setcounter{tocdepth}{3}





\begin{document}

\begin{titlepage}
    \begin{center}
     {\huge\bfseries 
    Bound and Inequalities     \\}
     % ----------------------------------------------------------------
     \vspace{1.5cm}
     {\Large\bfseries Hadar Tal}\\[5pt]
     hadar.tal@mail.huji.ac.il\\[14pt]
      % ----------------------------------------------------------------
     \vspace{2cm}
     {This paper is a summary of the educational materials and lectures from 
     \begin{itemize}
        \item \textbf{Wikipedia}
        \item \textbf{3Blue1Brown} YouTube channel
     \end{itemize}
     }

     \vfill
    {Winter 2024}
    \end{center}
\end{titlepage}


\frontmatter
\tableofcontents

% * * * * * * * * * * * * * * * * * * * * * * * * 
% * * * * * * * * * * * * * * * * * * * * * * * * 
% * * * * * * * * * * * * * * * * * * * * * * * * 
% * * * * * * * * * * * * * * * * * * * * * * * * 
% * * * * * * * * * * * * * * * * * * * * * * * * 
% * * * * * * * * * * * * * * * * * * * * * * * * 
% * * * * * * * * * * * * * * * * * * * * * * * * 
% * * * * * * * * * * * * * * * * * * * * * * * * 
% * * * * * * * * * * * * * * * * * * * * * * * * 
% * * * * * * * * * * * * * * * * * * * * * * * * 
% * * * * * * * * * * * * * * * * * * * * * * * * 
% * * * * * * * * * * * * * * * * * * * * * * * * 
% * * * * * * * * * * * * * * * * * * * * * * * * 
% * * * * * * * * * * * * * * * * * * * * * * * * 
% * * * * * * * * * * * * * * * * * * * * * * * * 
% * * * * * * * * * * * * * * * * * * * * * * * * 
% * * * * * * * * * * * * * * * * * * * * * * * * 
% * * * * * * * * * * * * * * * * * * * * * * * * 
% * * * * * * * * * * * * * * * * * * * * * * * * 

\mainmatter

\chapter{Bounds}

\chapter{Inequalities}

\begin{theorem}{\textbf{Cauchy-Schwarz Inequality}} \\
    Let $u, v$ be vectors of an inner product space. Then
    \begin{equation*}
        \left| \langle u, v \rangle \right|^2 \leq \left\| u \right\| \left\| v \right\|
    \end{equation*}
\end{theorem} 




\end{document}