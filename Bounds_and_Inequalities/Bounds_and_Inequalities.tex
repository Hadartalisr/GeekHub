\documentclass[11pt]{book} % or report
\usepackage{amsmath}
\usepackage{amsfonts}
\usepackage{amssymb}
\usepackage{geometry}
\geometry{a4paper, margin=1in}
\usepackage{graphicx}
\usepackage[hidelinks]{hyperref}
\usepackage{amsthm}
\usepackage{tikz}
\usepackage{subcaption}
\usetikzlibrary{positioning}
\usepackage{pgfplots} 
\usepackage[ruled,vlined]{algorithm2e} 
\usepackage{dsfont}
\usepackage{graphicx}
\usepackage{mathdesign}
\usepackage{float}
\usepackage{todonotes} 
\usepackage{empheq}
\usepackage{array}
\usepackage[ruled,vlined]{algorithm2e} 
\usepackage[many]{tcolorbox}    	



\setlength{\parindent}{0pt}

% \let\stdsection\section
% \renewcommand\section{\newpage\stdsection}

\newcommand\mycommfont[1]{\footnotesize\ttfamily\textcolor{blue}{#1}}
\newcommand\defeq{\stackrel{\mathclap{\normalfont\mbox{def}}}{=}}
\SetCommentSty{mycommfont}

\DeclareMathOperator*{\argmax}{argmax}
\DeclareMathOperator*{\argmin}{argmin}

\newtheorem{theorem}{Theorem}[section]
\newtheorem{lemma}{Lemma}[section]
\newtheorem{definition}{Definition}[section]
\newtheorem{corollary}{Corollary}[section]
\newtheorem{claim}{claim}[section]
\newtheorem{example}{Example}[section]


\newtheorem*{claim*}{Claim}
\newtheorem*{lemma*}{Lemma}
\newtheorem*{corollary*}{Corollary}
\newtheorem*{remark*}{Remark}
\newtheorem*{example*}{Example}
\newtheorem*{examples*}{Examples}
\newtheorem*{definition*}{Definition}



\setcounter{tocdepth}{3}


\newtcolorbox{boxA}{
    fontupper = \bf,
    boxrule = 1.5pt,
    colframe = black 
}


\begin{document}

\begin{titlepage}
    \begin{center}
     {\huge\bfseries 
    Bound, Equalities and Inequalities     \\}
     % ----------------------------------------------------------------
     \vspace{1.5cm}
     {\Large\bfseries Hadar Tal}\\[5pt]
     hadar.tal@mail.huji.ac.il\\[14pt]
      % ----------------------------------------------------------------
     \vspace{2cm}
     {This paper is a summary of the educational materials and lectures from 
     \begin{itemize}
        \item \textbf{Wikipedia}
        \item \textbf{3Blue1Brown} YouTube channel
     \end{itemize}
     }

     \vfill
    {Winter 2024}
    \end{center}
\end{titlepage}


\frontmatter
\tableofcontents


\mainmatter

% * * * * * * * * * * * * * * * * * * * * * * * * 
% * * * * * * * * * * * * * * * * * * * * * * * * 
% * * * * * * * * * * * * * * * * * * * * * * * * 
% * * * * * * * * * * * * * * * * * * * * * * * * 
% * * * * * * * * * * * * * * * * * * * * * * * * 
% * * * * * * * * * * * * * * * * * * * * * * * * 
% * * * * * * * * * * * * * * * * * * * * * * * * 
% * * * * * * * * * * * * * * * * * * * * * * * * 
% * * * * * * * * * * * * * * * * * * * * * * * * 
% * * * * * * * * * * * * * * * * * * * * * * * * 
% * * * * * * * * * * * * * * * * * * * * * * * * 
% * * * * * * * * * * * * * * * * * * * * * * * * 
% * * * * * * * * * * * * * * * * * * * * * * * * 
% * * * * * * * * * * * * * * * * * * * * * * * * 
% * * * * * * * * * * * * * * * * * * * * * * * * 
% * * * * * * * * * * * * * * * * * * * * * * * * 
% * * * * * * * * * * * * * * * * * * * * * * * * 
% * * * * * * * * * * * * * * * * * * * * * * * * 
% * * * * * * * * * * * * * * * * * * * * * * * * 
\chapter{Bounds}


% * * * * * * * * * * * * * * * * * * * * * * * * 
% * * * * * * * * * * * * * * * * * * * * * * * * 
% * * * * * * * * * * * * * * * * * * * * * * * * 
% * * * * * * * * * * * * * * * * * * * * * * * * 
% * * * * * * * * * * * * * * * * * * * * * * * * 
% * * * * * * * * * * * * * * * * * * * * * * * * 
% * * * * * * * * * * * * * * * * * * * * * * * * 
% * * * * * * * * * * * * * * * * * * * * * * * * 
% * * * * * * * * * * * * * * * * * * * * * * * * 
% * * * * * * * * * * * * * * * * * * * * * * * * 
% * * * * * * * * * * * * * * * * * * * * * * * * 
% * * * * * * * * * * * * * * * * * * * * * * * * 
% * * * * * * * * * * * * * * * * * * * * * * * * 
% * * * * * * * * * * * * * * * * * * * * * * * * 
% * * * * * * * * * * * * * * * * * * * * * * * * 
% * * * * * * * * * * * * * * * * * * * * * * * * 
% * * * * * * * * * * * * * * * * * * * * * * * * 
% * * * * * * * * * * * * * * * * * * * * * * * * 
% * * * * * * * * * * * * * * * * * * * * * * * * 
\chapter{Equalities}

\section{Properties of Binomial Coefficients}

\subsection{Symmetry Rule for Binomial Coefficients}

\begin{boxA}
    \begin{theorem}
        For all $n, k \in \mathbb{N}$, the following holds
        \begin{equation}
            \binom{n}{k} = \binom{n}{n-k}
        \end{equation}
    \end{theorem}
\end{boxA}

\begin{proof}
    The proof is by definition of the binomial coefficient
    \begin{equation}
        \binom{n}{k} = \frac{n!}{k!(n-k)!}
    \end{equation}
    and the symmetry of the factorial function
    \begin{equation}
        n! = n \cdot (n-1) \cdot \ldots \cdot 2 \cdot 1 = 1 \cdot 2 \cdot \ldots \cdot (n-1) \cdot n = n!
    \end{equation}
    which implies that
    \begin{equation}
        \binom{n}{k} = \frac{n!}{k!(n-k)!} = \frac{n!}{(n-k)!k!} = \binom{n}{n-k}
    \end{equation}
\end{proof}

\begin{example*}
    The symmetry rule for binomial coefficients states that the number of ways to choose $k$ elements out of $n$ is the same as the number of ways to choose $n-k$ elements out of $n$.
\end{example*}


\subsection{Pascal's Rule for Binomial Coefficients}

\begin{boxA}
    \begin{theorem}
        For all $n, k \in \mathbb{N}$, the following holds
        \begin{equation}
            \binom{n}{k} = \binom{n-1}{k} + \binom{n-1}{k-1}
        \end{equation}
    \end{theorem}
\end{boxA}

\begin{proof}
    \begin{equation}
        \binom{n}{k} = \frac{n!}{k!(n-k)!} = \frac{(n-1)!}{k!(n-1-k)!} + \frac{(n-1)!}{(k-1)!(n-1-(k-1))!} = \binom{n-1}{k} + \binom{n-1}{k-1}
    \end{equation}
\end{proof}

\begin{example*}
    choosing a team of k players from n candidates: you can either include a specific player in your team and choose the rest 
    k-1 players from the remaining n-1 candidates, or not include that specific player, thus choosing all k players from the remaining n-1 candidates.
\end{example*}

\begin{figure}[H]
    \centering
    \includegraphics[width=0.4\textwidth]{figs/pascals_triangle.jpeg}
    \caption{Pascal's Triangle}
\end{figure}

\subsection{Sum of Binomial Coefficients over Lower Index}

\begin{boxA}
    \begin{theorem}
        For all $n\in \mathbb{N}$, the following holds
        \begin{equation}
            \sum_{i=0}^{n} \binom{n}{i} = 2^n
        \end{equation}
    \end{theorem}
\end{boxA}

\begin{proof}
    The proof is by induction on $n$. For $n=0$, the base case is
    \begin{equation}
        \sum_{i=0}^{0} \binom{0}{i} = \binom{0}{0} = 1 = 2^0
    \end{equation}
    For the induction step, assume that the theorem holds for $n=k$. Then
    \begin{equation}
        \sum_{i=0}^{k+1} \binom{k+1}{i} = \sum_{i=0}^{k+1} \left( \binom{k}{i} + \binom{k}{i-1} \right) = \sum_{i=0}^{k} \binom{k}{i} + \sum_{i=0}^{k} \binom{k}{i-1} = 2^k + 2^k = 2^{k+1}
    \end{equation}
\end{proof}

\begin{example*}
    The sum of binomial coefficients over the lower index is the same as counting all the subsets of a set of size $n$ which is $2^n$.
\end{example*}


\subsection{Factors of Binomial Coefficient}

\begin{boxA}
    \begin{theorem}
        For all $r \in \mathbb{R}, k \in \mathbb{Z}$, the following holds:
        \begin{equation}
            k \binom{r}{k} = r \binom{r-1}{k-1}
        \end{equation}
    \end{theorem}
\end{boxA}

\begin{proof}
    By definition, the binomial coefficient $\binom{r}{k}$ is given by
    \begin{equation}
        \binom{r}{k} = \frac{r!}{k!(r-k)!}.
    \end{equation}
    Multiplying both sides by $k$, we have
    \begin{equation}
        k \binom{r}{k} = k \frac{r!}{k!(r-k)!} = \frac{r!}{(k-1)!(r-k)!} = r \frac{(r-1)!}{(k-1)!(r-k)!} = r \binom{r-1}{k-1}.
    \end{equation}
\end{proof}

\begin{example*}
    Forming a committee of $k$ members from a group of $r$ individuals, with one member to be chosen as chairperson.
\end{example*}

\subsection{Increasing Sum of Binomial Coefficients}

\begin{boxA}
    \begin{theorem}
        For all $n\in \mathbb{N}$, the following holds
        \begin{equation}
            \sum_{k=0}^{n} k \binom{n}{k} = n 2^{n-1}
        \end{equation}
    \end{theorem}
\end{boxA}

\begin{proof}
    \begin{equation}
        \sum_{k=0}^{n} k \binom{n}{k} = \sum_{k=1}^{n} k \binom{n}{k} = \sum_{k=1}^{n} n \binom{n-1}{k-1} = n \sum_{k=1}^{n} \binom{n-1}{k-1} = n \sum_{k=0}^{n-1} \binom{n-1}{k} = n \cdot 2^{n-1}
    \end{equation}
\end{proof}

% * * * * * * * * * * * * * * * * * * * * * * * * 
% * * * * * * * * * * * * * * * * * * * * * * * * 
% * * * * * * * * * * * * * * * * * * * * * * * * 
% * * * * * * * * * * * * * * * * * * * * * * * * 
% * * * * * * * * * * * * * * * * * * * * * * * * 
% * * * * * * * * * * * * * * * * * * * * * * * * 
% * * * * * * * * * * * * * * * * * * * * * * * * 
% * * * * * * * * * * * * * * * * * * * * * * * * 
% * * * * * * * * * * * * * * * * * * * * * * * * 
% * * * * * * * * * * * * * * * * * * * * * * * * 
% * * * * * * * * * * * * * * * * * * * * * * * * 
% * * * * * * * * * * * * * * * * * * * * * * * * 
% * * * * * * * * * * * * * * * * * * * * * * * * 
% * * * * * * * * * * * * * * * * * * * * * * * * 
% * * * * * * * * * * * * * * * * * * * * * * * * 
% * * * * * * * * * * * * * * * * * * * * * * * * 
% * * * * * * * * * * * * * * * * * * * * * * * * 
% * * * * * * * * * * * * * * * * * * * * * * * * 
% * * * * * * * * * * * * * * * * * * * * * * * * 
\chapter{Inequalities}

\begin{theorem}{\textbf{Cauchy-Schwarz Inequality}} \\
    Let $u, v$ be vectors of an inner product space. Then
    \begin{equation*}
        \left| \langle u, v \rangle \right|^2 \leq \left\| u \right\| \left\| v \right\|
    \end{equation*}
\end{theorem} 




\end{document}