\documentclass[a4 paper]{article}
\usepackage[inner=2.0cm,outer=2.0cm,top=2.5cm,bottom=2.5cm]{geometry}
\usepackage{setspace}
\usepackage[rgb]{xcolor}
\usepackage{verbatim}
\usepackage{subcaption}
\usepackage{amsgen,amsmath,amstext,amsbsy,amsopn,tikz,amssymb}
\usepackage{fancyhdr}
\usepackage[colorlinks=true, urlcolor=blue,  linkcolor=blue, citecolor=blue]{hyperref}
\usepackage[colorinlistoftodos]{todonotes}
\usepackage{rotating}
\usepackage{tikz}
\usetikzlibrary{decorations.pathreplacing} 
\usetikzlibrary{positioning}
\usepackage{float}
\usepackage{minted}
\usepackage{amsthm}
\usepackage{amsmath}
\usepackage{amsfonts}
\usepackage{amssymb}
\usepackage{geometry}
\geometry{a4paper, margin=1in}
\usepackage{graphicx}
% \usepackage[hidelinks]{hyperref}
\usepackage{dsfont}
\usepackage{graphicx}
\usepackage{mathdesign}
\usepackage{float}
\usepackage{todonotes} 
\usepackage{empheq}
\usepackage{array}
\usepackage[ruled,vlined]{algorithm2e} 
\usepackage[many]{tcolorbox}    	% for COLORED BOXES (tikz and xcolor included)
\usepackage{bookmark}

\usepackage{pgfplots} 
\pgfplotsset{compat=1.18}


\newtcolorbox{boxA}{
    fontupper = \bf,
    boxrule = 1.5pt,
    colframe = black % frame color
}


\setlength{\parindent}{0pt}
\numberwithin{equation}{section}


\newcommand\mycommfont[1]{\footnotesize\ttfamily\textcolor{blue}{#1}}
\newcommand\defeq{\stackrel{\mathclap{\normalfont\mbox{def}}}{=}}
\SetCommentSty{mycommfont}

\DeclareMathOperator*{\argmax}{argmax}
\DeclareMathOperator*{\argmin}{argmin}



\newtheoremstyle{boldStyle}%                % Name
  {}%                                     % Space above
  {}%                                     % Space below
  {\itshape}%                                     % Body font
  {}%                                     % Indent amount
  {\bfseries}%                            % Theorem head font
  {}%                                    % Punctuation after theorem head
  {\newline}                              % Space after theorem head, new line
  {\thmname{#1}\thmnumber{ #2}\thmnote{ (#3)}}%                                     % Theorem head spec (can be left empty, meaning `normal')


\theoremstyle{boldStyle}
\newtheorem{example}{Example}[section]
\newtheorem{question}{Question}[section]
\newtheorem{answer}{Answer}[section]


\newtheorem*{claim*}{Claim}
\newtheorem*{lemma*}{Lemma}
\newtheorem*{corollary*}{Corollary}
\newtheorem*{remark*}{Remark}
\newtheorem*{example*}{Example}
\newtheorem*{examples*}{Examples}
\newtheorem*{definition*}{Definition}
\newtheorem*{question*}{Question}
\newtheorem*{answer*}{Answer}


\definecolor{blueColor}{rgb}{0, 0.611, 0.98} 
\newtheoremstyle{boldBlueStyle}%                % Name
  {}%                                          % Space above
  {}%                                          % Space below
  {\itshape}%                                  % Body font
  {}%                                          % Indent amount
  {\color{blueColor}\bfseries}%          % Theorem head font in red
  {}%                                          % Punctuation after theorem head
  {\newline}%                                  % Space after theorem head, new line
  {\thmname{#1}\thmnumber{ #2}\thmnote{ (#3)}} % Theorem head spec (can be left empty, meaning `normal')

\theoremstyle{boldBlueStyle}
\newtheorem{lemma}{Lemma}[section]
\newtheorem{corollary}{Corollary}[section]
\newtheorem{claim}{claim}[section]
\newtheorem{proposition}{Proposition}[section]



\definecolor{purpleColor}{rgb}{0.59, 0.223, 0.6} 
\newtheoremstyle{boldPurpleStyle}%                % Name
  {}%                                          % Space above
  {}%                                          % Space below
  {\itshape}%                                  % Body font
  {}%                                          % Indent amount
  {\color{purpleColor}\bfseries}%          % Theorem head font in red
  {}%                                          % Punctuation after theorem head
  {\newline}%                                  % Space after theorem head, new line
  {\thmname{#1}\thmnumber{ #2}\thmnote{ (#3)}} % Theorem head spec (can be left empty, meaning `normal')

\theoremstyle{boldPurpleStyle}
\newtheorem{theorem}{Theorem}[section]


\definecolor{redColor}{rgb}{1, 0.219, 0.219} 
\newtheoremstyle{boldRedStyle}%                % Name
  {}%                                          % Space above
  {}%                                          % Space below
  {\itshape}%                                  % Body font
  {}%                                          % Indent amount
  {\color{redColor}\bfseries}%          % Theorem head font in red
  {}%                                          % Punctuation after theorem head
  {\newline}%                                  % Space after theorem head, new line
  {\thmname{#1}\thmnumber{ #2}\thmnote{ (#3)}} % Theorem head spec (can be left empty, meaning `normal')

\theoremstyle{boldRedStyle}
\newtheorem{definition}{Definition}[section]




%\usetikzlibrary{through,backgrounds}
\hypersetup{%
pdfauthor={Ashudeep Singh},%
pdftitle={Homework},%
pdfkeywords={Tikz,latex,bootstrap,uncertaintes},%
pdfcreator={PDFLaTeX},%
pdfproducer={PDFLaTeX},%
}
%\usetikzlibrary{shadows}
% \usepackage[francais]{babel}
\usepackage{booktabs}
\input{../../Latex_Utils/macros.tex}

\setlength{\parindent}{0pt}


\begin{document}
\homework{67939 - Topics in Learning Theory}{Exercise 1}{Due: 16/06/24}{Prof. Amit Daniely}{}{Hadar Tal}{}



\section*{Exercise 1}
The moment generating function (MGF) of a random variable \(X\) is \(M_X(\lambda) = \mathbb{E}[e^{\lambda X}]\). Assume that \(M_X\) is defined for any \(\lambda\) in a non-empty segment \((-a, a)\). Show that

\begin{enumerate}
    \item \(M_X^{(k)}(0) = \mathbb{E}[X^k]\)
    \item Show that for a centered Gaussian \(X\) with variance \(\sigma^2\), \(M_X(\lambda) = e^{\frac{\lambda^2 \sigma^2}{2}}\). In other words, being \(\sigma\)-SubGaussian is equivalent to having MGF that is bounded by the MGF of a centered Gaussian with variance \(\sigma^2\).
    \item Show that if \(X\) is uniform over \([a, b]\) then \(M_X(\lambda) = \frac{e^{\lambda b} - e^{\lambda a}}{\lambda (b - a)}\).
\end{enumerate}

\section*{Exercise 2}
\begin{enumerate}
    \item Show that if \(X_i\) is \(\sigma_i\)-SubGaussian for \(i = 1, 2\) then \(X_1 + X_2\) is \((\sigma_1 + \sigma_2)\)-SubGaussian\footnote{Use the Hölder inequality \((\mathbb{E}[XY] \leq (\mathbb{E}[X^p])^{1/p} (\mathbb{E}[Y^q])^{1/q} \text{ if } \frac{1}{p} + \frac{1}{q} = 1 \text{ and } p, q \geq 0)\) on \(\mathbb{E}[e^{\lambda (X - \mathbb{E}[X])} e^{\lambda (Y - \mathbb{E}[Y])}]\)}.
    \item For a sub-Gaussian random variable \(X\), define \(\|X\|_{vp}\) as the minimal \(\sigma\) for which \(X\) is \(\sigma\)-SubGaussian. Show that \(\|\cdot\|_{vp}\) is a norm on the space of centered sub-Gaussian random variables. This norm is called the Proxy Variance norm and \(\|X\|_{vp}\) is called the optimal proxy variance of \(X\).
\end{enumerate}

\section*{Exercise 3}
\begin{enumerate}
    \item Let \(X\) be a \(\sigma\)-SubGaussian random variable. Show that \(2\sigma \geq \sqrt{\mathrm{var}(X)}\).
    \item If \(\|X\|_{vp} = \sqrt{\mathrm{var}(X)}\), then \(X\) is called strictly sub-Gaussian. Show that if \(X\) is uniform on \(\{-1, 1\}\), then it is strictly sub-Gaussian. Conclude that the bound in Hoeffding's lemma is optimal.
    \item Show that a linear combination of independent strictly sub-Gaussians is strictly sub-Gaussian.
    \item Show that for any \(M \geq 1\), there is a random variable \(X\) with \(\mathrm{var}(X) = 1\) and \(\|X\|_{vp} = M\).
\end{enumerate}

\section*{Exercise 4}
Show that there is a universal constant \(C > 0\) for which the following holds. If \(X\) is a random variable such that for any \(t \geq 0\),

\[
\Pr(X - \mathbb{E}[X] \geq t) \leq e^{-\frac{t^2}{2\sigma^2}} \quad \text{and} \quad \Pr(X - \mathbb{E}[X] \leq -t) \leq e^{-\frac{t^2}{2\sigma^2}}
\]

then \(X\) is \((C\sigma)\)-SubGaussian\footnote{Hint: You may use the fact that for a non-negative random variable \(Y\), \(\mathbb{E}[Y] = \int_0^\infty \Pr(Y \geq x)dx\)}.


\end{document}