\documentclass[a4 paper]{article}
\usepackage[inner=2.0cm,outer=2.0cm,top=2.5cm,bottom=2.5cm]{geometry}
\usepackage{setspace}
\usepackage[rgb]{xcolor}
\usepackage{verbatim}
\usepackage{subcaption}
\usepackage{amsgen,amsmath,amstext,amsbsy,amsopn,tikz,amssymb}
\usepackage{fancyhdr}
\usepackage[colorlinks=true, urlcolor=blue,  linkcolor=blue, citecolor=blue]{hyperref}
\usepackage[colorinlistoftodos]{todonotes}
\usepackage{rotating}
\usepackage{tikz}
\usetikzlibrary{decorations.pathreplacing} 
\usetikzlibrary{positioning}
\usepackage{float}
\usepackage{minted}
\usepackage{amsthm}
\usepackage{amsmath}
\usepackage{amsfonts}
\usepackage{amssymb}
\usepackage{geometry}
\geometry{a4paper, margin=1in}
\usepackage{graphicx}
% \usepackage[hidelinks]{hyperref}
\usepackage{dsfont}
\usepackage{graphicx}
\usepackage{mathdesign}
\usepackage{float}
\usepackage{todonotes} 
\usepackage{empheq}
\usepackage{array}
\usepackage[ruled,vlined]{algorithm2e} 
\usepackage[many]{tcolorbox}    	% for COLORED BOXES (tikz and xcolor included)
\usepackage{bookmark}

\usepackage{pgfplots} 
\pgfplotsset{compat=1.18}


\newtcolorbox{boxA}{
    fontupper = \bf,
    boxrule = 1.5pt,
    colframe = black % frame color
}


\setlength{\parindent}{0pt}
\numberwithin{equation}{section}


\newcommand\mycommfont[1]{\footnotesize\ttfamily\textcolor{blue}{#1}}
\newcommand\defeq{\stackrel{\mathclap{\normalfont\mbox{def}}}{=}}
\SetCommentSty{mycommfont}

\DeclareMathOperator*{\argmax}{argmax}
\DeclareMathOperator*{\argmin}{argmin}



\newtheoremstyle{boldStyle}%                % Name
  {}%                                     % Space above
  {}%                                     % Space below
  {\itshape}%                                     % Body font
  {}%                                     % Indent amount
  {\bfseries}%                            % Theorem head font
  {}%                                    % Punctuation after theorem head
  {\newline}                              % Space after theorem head, new line
  {\thmname{#1}\thmnumber{ #2}\thmnote{ (#3)}}%                                     % Theorem head spec (can be left empty, meaning `normal')


\theoremstyle{boldStyle}
\newtheorem{example}{Example}[section]
\newtheorem{question}{Question}[section]
\newtheorem{answer}{Answer}[section]


\newtheorem*{claim*}{Claim}
\newtheorem*{lemma*}{Lemma}
\newtheorem*{corollary*}{Corollary}
\newtheorem*{remark*}{Remark}
\newtheorem*{example*}{Example}
\newtheorem*{examples*}{Examples}
\newtheorem*{definition*}{Definition}
\newtheorem*{question*}{Question}
\newtheorem*{answer*}{Answer}


\definecolor{blueColor}{rgb}{0, 0.611, 0.98} 
\newtheoremstyle{boldBlueStyle}%                % Name
  {}%                                          % Space above
  {}%                                          % Space below
  {\itshape}%                                  % Body font
  {}%                                          % Indent amount
  {\color{blueColor}\bfseries}%          % Theorem head font in red
  {}%                                          % Punctuation after theorem head
  {\newline}%                                  % Space after theorem head, new line
  {\thmname{#1}\thmnumber{ #2}\thmnote{ (#3)}} % Theorem head spec (can be left empty, meaning `normal')

\theoremstyle{boldBlueStyle}
\newtheorem{lemma}{Lemma}[section]
\newtheorem{corollary}{Corollary}[section]
\newtheorem{claim}{claim}[section]
\newtheorem{proposition}{Proposition}[section]



\definecolor{purpleColor}{rgb}{0.59, 0.223, 0.6} 
\newtheoremstyle{boldPurpleStyle}%                % Name
  {}%                                          % Space above
  {}%                                          % Space below
  {\itshape}%                                  % Body font
  {}%                                          % Indent amount
  {\color{purpleColor}\bfseries}%          % Theorem head font in red
  {}%                                          % Punctuation after theorem head
  {\newline}%                                  % Space after theorem head, new line
  {\thmname{#1}\thmnumber{ #2}\thmnote{ (#3)}} % Theorem head spec (can be left empty, meaning `normal')

\theoremstyle{boldPurpleStyle}
\newtheorem{theorem}{Theorem}[section]


\definecolor{redColor}{rgb}{1, 0.219, 0.219} 
\newtheoremstyle{boldRedStyle}%                % Name
  {}%                                          % Space above
  {}%                                          % Space below
  {\itshape}%                                  % Body font
  {}%                                          % Indent amount
  {\color{redColor}\bfseries}%          % Theorem head font in red
  {}%                                          % Punctuation after theorem head
  {\newline}%                                  % Space after theorem head, new line
  {\thmname{#1}\thmnumber{ #2}\thmnote{ (#3)}} % Theorem head spec (can be left empty, meaning `normal')

\theoremstyle{boldRedStyle}
\newtheorem{definition}{Definition}[section]




%\usetikzlibrary{through,backgrounds}
\hypersetup{%
pdfauthor={Ashudeep Singh},%
pdftitle={Homework},%
pdfkeywords={Tikz,latex,bootstrap,uncertaintes},%
pdfcreator={PDFLaTeX},%
pdfproducer={PDFLaTeX},%
}
%\usetikzlibrary{shadows}
% \usepackage[francais]{babel}
\usepackage{booktabs}
\input{../../Latex_Utils/macros.tex}

\setlength{\parindent}{0pt}


\begin{document}
\homework{67939 - Topics in Learning Theory}{Exercise 1}{Due: 16/06/24}{Prof. Amit Daniely}{}{Hadar Tal}{}



\section*{Exercise 1}
The moment generating function (MGF) of a random variable \(X\) is \(M_X(\lambda) = \mathbb{E}[e^{\lambda X}]\). 
Assume that \(M_X\) is defined for any \(\lambda\) in a non-empty segment \((-a, a)\). Show that

\begin{enumerate}
\item \(M_X^{(k)}(0) = \mathbb{E}[X^k]\)

\bigbreak
    
Using the definition of the moment-generating function, we can write:
\[
M_X^{(k)}(t) = \frac{d^k}{d\lambda^k} \mathbb{E}[e^{\lambda X}] 
\]

Using the power series expansion of the exponential function
\[
e^x = \sum_{k=0}^{\infty} \frac{x^k}{k!}
\]

we can write
\[
M_X^{(k)}(t) = \frac{d^k}{d\lambda^k} \mathbb{E} \left( \sum_{m=0}^{\infty} \frac{\lambda^m X^m}{m!} \right)
\]

Because the expected value is a linear operator, we have:
\[
M_X^{(k)}(t) = \frac{d^k}{d\lambda^k} \sum_{m=0}^{\infty} \mathbb{E} \left( \frac{\lambda^m X^m}{m!} \right) 
  = \sum_{m=0}^{\infty} \frac{d^k}{d\lambda^k} \left( \frac{\lambda^m}{m!} \right) \mathbb{E}[X^m]
\]

Using the \(k\)-th derivative of the \(m\)-th power
\[
\frac{d^k}{d\lambda^k} \lambda^m = 
\begin{cases} 
\tilde{m}^k \lambda^{m-k}, & \text{if } k \leq m \\
0, & \text{if } k > m
\end{cases} 
\]

when 
\[
\tilde{m}^k  = \prod_{i=0}^{k-1} (m-i) = \frac{m!}{(m-k)!}
\]

then we have
\begin{align*}
M_X^{(k)}(\lambda) &= \sum_{m=0}^{\infty} \frac{d^k}{d\lambda^k} \left( \frac{\lambda^m}{m!} \right) \mathbb{E}[X^m] = \sum_{m=k}^{\infty} \frac{\tilde{m^k} \lambda^{m-k}}{m!} \mathbb{E}[X^m] 
  = \sum_{m=k}^{\infty} \frac{m! \lambda^{m-k}}{(m-k)! m!} \mathbb{E}[X^m] \\
  &= \sum_{m=k}^{\infty} \frac{\lambda^{m-k}}{(m-k)!} \mathbb{E}[X^m] = \frac{t^{n-n}}{(n-n)!} \mathbb{E}[X^n] + \sum_{m=k+1}^{\infty} \frac{\lambda^{m-k}}{(m-k)!} \mathbb{E}[X^m] \\
  &= \mathbb{E}[X^k] + \sum_{m=k+1}^{\infty} \frac{\lambda^{m-k}}{(m-k)!} \mathbb{E}[X^m]
\end{align*}

Setting \(\lambda = 0\) in the above equation, we get
\[
M_X^{(k)}(0) = \mathbb{E}[X^k] + \sum_{m=k+1}^{\infty} \frac{0^{m-k}}{(m-k)!} \mathbb{E}[X^m] = \mathbb{E}[X^k]
\]
which completes the proof.


    
\newpage
\item Show that for a centered Gaussian \(X\) with variance \(\sigma^2\), \(M_X(\lambda) = e^{\frac{\lambda^2 \sigma^2}{2}}\). 
In other words, being \(\sigma\)-SubGaussian is equivalent to having MGF that is bounded by the MGF of a centered Gaussian with variance \(\sigma^2\).
    



Let \(X\) be a centered Gaussian random variable with mean \(\mathbb{E}[X] = 0\) and variance \(\mathrm{var}(X) = \sigma^2\). The moment generating function (MGF) of \(X\) is defined as:
\[
M_X(\lambda) = \mathbb{E}[e^{\lambda X}].
\]

Since \(X\) is Gaussian, \(X\) has the probability density function:
\[
f_X(x) = \frac{1}{\sqrt{2 \pi \sigma^2}} e^{-\frac{x^2}{2\sigma^2}}
\]

Therefore, the MGF \(M_X(\lambda)\) is:
\[
M_X(\lambda) = \int_{-\infty}^{\infty} e^{\lambda x} f_X(x) \, dx = \int_{-\infty}^{\infty} e^{\lambda x} \frac{1}{\sqrt{2 \pi \sigma^2}} e^{-\frac{x^2}{2\sigma^2}} \, dx
\]

Combining the exponents, we get:
\[
M_X(\lambda) = \int_{-\infty}^{\infty} \frac{1}{\sqrt{2 \pi \sigma^2}} e^{\lambda x - \frac{x^2}{2\sigma^2}} \, dx
\]

Completing the square in the exponent:
\[
\lambda x - \frac{x^2}{2\sigma^2} = -\frac{1}{2\sigma^2} \left(x^2 - 2\sigma^2 \lambda x \right) = -\frac{1}{2\sigma^2} \left(x^2 - 2\sigma^2 \lambda x + \sigma^4 \lambda^2 - \sigma^4 \lambda^2 \right) = -\frac{1}{2\sigma^2} \left((x - \sigma^2 \lambda)^2 - \sigma^4 \lambda^2 \right).
\]

Thus, the integral becomes:
\[
M_X(\lambda) = \int_{-\infty}^{\infty} \frac{1}{\sqrt{2 \pi \sigma^2}} e^{-\frac{1}{2\sigma^2} (x - \sigma^2 \lambda)^2} e^{\frac{\sigma^2 \lambda^2}{2}} \, dx
\]

Since the first term inside the integral is a normal distribution that integrates to 1, we get:
\[
M_X(\lambda) = e^{\frac{\sigma^2 \lambda^2}{2}} \int_{-\infty}^{\infty} \frac{1}{\sqrt{2 \pi \sigma^2}} e^{-\frac{1}{2\sigma^2} (x - \sigma^2 \lambda)^2} \, dx = e^{\frac{\sigma^2 \lambda^2}{2}}
\]

Therefore, the MGF of \(X\) is:
\[
M_X(\lambda) = e^{\frac{\lambda^2 \sigma^2}{2}}
\]

This shows that being \(\sigma\)-SubGaussian is equivalent to having an MGF that is bounded by the MGF of a centered Gaussian with variance \(\sigma^2\).










    
    
\newpage
\item Show that if \(X\) is uniform over \([a, b]\) then \(M_X(\lambda) = \frac{e^{\lambda b} - e^{\lambda a}}{\lambda (b - a)}\).



\end{enumerate}
















\newpage
\section*{Exercise 2}
\begin{enumerate}
\item Show that if \(X_i\) is \(\sigma_i\)-SubGaussian for \(i = 1, 2\) then \(X_1 + X_2\) is \((\sigma_1 + \sigma_2)\)-SubGaussian
\footnote{Use the Hölder inequality \((\mathbb{E}[XY] \leq (\mathbb{E}[X^p])^{1/p} (\mathbb{E}[Y^q])^{1/q} \text{ if } \frac{1}{p} + \frac{1}{q} = 1 \text{ and } p, q \geq 0)\) 
on \(\mathbb{E}[e^{\lambda (X - \mathbb{E}[X])} e^{\lambda (Y - \mathbb{E}[Y])}]\)}.



\newpage
\item For a sub-Gaussian random variable \(X\), define \(\|X\|_{vp}\) as the minimal \(\sigma\) for which \(X\) is \(\sigma\)-SubGaussian. 
Show that \(\|\cdot\|_{vp}\) is a norm on the space of centered sub-Gaussian random variables. This norm is called the Proxy Variance norm and \(\|X\|_{vp}\) 
is called the optimal proxy variance of \(X\).



\end{enumerate}

\newpage
\section*{Exercise 3}
\begin{enumerate}
\item Let \(X\) be a \(\sigma\)-SubGaussian random variable. Show that
\footnote{Hint: You can use the fact that for twice differentiable \( f \) and \( g \), we have that if \( f(0) = g(0) \), \( f'(0) = g'(0) \) and \( f(x) \leq g(x) \) then \( f''(0) \leq g''(0) \)}
\(2\sigma \geq \sqrt{\mathrm{var}(X)}\).



\newpage
\item If \(\|X\|_{vp} = \sqrt{\mathrm{var}(X)}\), then \(X\) is called strictly sub-Gaussian. 
Show that if \(X\) is uniform on \(\{-1, 1\}\), then it is strictly sub-Gaussian. Conclude that the bound in Hoeffding's lemma is optimal.



\newpage
\item Show that a linear combination of independent strictly sub-Gaussians is strictly sub-Gaussian.



\newpage
\item Show that for any \(M \geq 1\), there is a random variable \(X\) with \(\mathrm{var}(X) = 1\) and \(\|X\|_{vp} = M\).
\end{enumerate}

\newpage
\section*{Exercise 4}
Show that there is a universal constant \(C > 0\) for which the following holds. If \(X\) is a random variable such that for any \(t \geq 0\),

\[
\Pr(X - \mathbb{E}[X] \geq t) \leq e^{-\frac{t^2}{2\sigma^2}} \quad \text{and} \quad \Pr(X - \mathbb{E}[X] \leq -t) \leq e^{-\frac{t^2}{2\sigma^2}}
\]

then \(X\) is \((C\sigma)\)-SubGaussian\footnote{Hint: You may use the fact that for a non-negative random variable \(Y\), \(\mathbb{E}[Y] = \int_0^\infty \Pr(Y \geq x)dx\)}.


\end{document}